% Options for packages loaded elsewhere
\PassOptionsToPackage{unicode}{hyperref}
\PassOptionsToPackage{hyphens}{url}
\PassOptionsToPackage{dvipsnames,svgnames,x11names}{xcolor}
%
\documentclass[
  12pt,
]{article}

\usepackage{amsmath,amssymb}
\usepackage{iftex}
\ifPDFTeX
  \usepackage[T1]{fontenc}
  \usepackage[utf8]{inputenc}
  \usepackage{textcomp} % provide euro and other symbols
\else % if luatex or xetex
  \usepackage{unicode-math}
  \defaultfontfeatures{Scale=MatchLowercase}
  \defaultfontfeatures[\rmfamily]{Ligatures=TeX,Scale=1}
\fi
\usepackage{lmodern}
\ifPDFTeX\else  
    % xetex/luatex font selection
\fi
% Use upquote if available, for straight quotes in verbatim environments
\IfFileExists{upquote.sty}{\usepackage{upquote}}{}
\IfFileExists{microtype.sty}{% use microtype if available
  \usepackage[]{microtype}
  \UseMicrotypeSet[protrusion]{basicmath} % disable protrusion for tt fonts
}{}
\makeatletter
\@ifundefined{KOMAClassName}{% if non-KOMA class
  \IfFileExists{parskip.sty}{%
    \usepackage{parskip}
  }{% else
    \setlength{\parindent}{0pt}
    \setlength{\parskip}{6pt plus 2pt minus 1pt}}
}{% if KOMA class
  \KOMAoptions{parskip=half}}
\makeatother
\usepackage{xcolor}
\usepackage[margin=1in]{geometry}
\setlength{\emergencystretch}{3em} % prevent overfull lines
\setcounter{secnumdepth}{5}
% Make \paragraph and \subparagraph free-standing
\makeatletter
\ifx\paragraph\undefined\else
  \let\oldparagraph\paragraph
  \renewcommand{\paragraph}{
    \@ifstar
      \xxxParagraphStar
      \xxxParagraphNoStar
  }
  \newcommand{\xxxParagraphStar}[1]{\oldparagraph*{#1}\mbox{}}
  \newcommand{\xxxParagraphNoStar}[1]{\oldparagraph{#1}\mbox{}}
\fi
\ifx\subparagraph\undefined\else
  \let\oldsubparagraph\subparagraph
  \renewcommand{\subparagraph}{
    \@ifstar
      \xxxSubParagraphStar
      \xxxSubParagraphNoStar
  }
  \newcommand{\xxxSubParagraphStar}[1]{\oldsubparagraph*{#1}\mbox{}}
  \newcommand{\xxxSubParagraphNoStar}[1]{\oldsubparagraph{#1}\mbox{}}
\fi
\makeatother


\providecommand{\tightlist}{%
  \setlength{\itemsep}{0pt}\setlength{\parskip}{0pt}}\usepackage{longtable,booktabs,array}
\usepackage{calc} % for calculating minipage widths
% Correct order of tables after \paragraph or \subparagraph
\usepackage{etoolbox}
\makeatletter
\patchcmd\longtable{\par}{\if@noskipsec\mbox{}\fi\par}{}{}
\makeatother
% Allow footnotes in longtable head/foot
\IfFileExists{footnotehyper.sty}{\usepackage{footnotehyper}}{\usepackage{footnote}}
\makesavenoteenv{longtable}
\usepackage{graphicx}
\makeatletter
\newsavebox\pandoc@box
\newcommand*\pandocbounded[1]{% scales image to fit in text height/width
  \sbox\pandoc@box{#1}%
  \Gscale@div\@tempa{\textheight}{\dimexpr\ht\pandoc@box+\dp\pandoc@box\relax}%
  \Gscale@div\@tempb{\linewidth}{\wd\pandoc@box}%
  \ifdim\@tempb\p@<\@tempa\p@\let\@tempa\@tempb\fi% select the smaller of both
  \ifdim\@tempa\p@<\p@\scalebox{\@tempa}{\usebox\pandoc@box}%
  \else\usebox{\pandoc@box}%
  \fi%
}
% Set default figure placement to htbp
\def\fps@figure{htbp}
\makeatother
% definitions for citeproc citations
\NewDocumentCommand\citeproctext{}{}
\NewDocumentCommand\citeproc{mm}{%
  \begingroup\def\citeproctext{#2}\cite{#1}\endgroup}
\makeatletter
 % allow citations to break across lines
 \let\@cite@ofmt\@firstofone
 % avoid brackets around text for \cite:
 \def\@biblabel#1{}
 \def\@cite#1#2{{#1\if@tempswa , #2\fi}}
\makeatother
\newlength{\cslhangindent}
\setlength{\cslhangindent}{1.5em}
\newlength{\csllabelwidth}
\setlength{\csllabelwidth}{3em}
\newenvironment{CSLReferences}[2] % #1 hanging-indent, #2 entry-spacing
 {\begin{list}{}{%
  \setlength{\itemindent}{0pt}
  \setlength{\leftmargin}{0pt}
  \setlength{\parsep}{0pt}
  % turn on hanging indent if param 1 is 1
  \ifodd #1
   \setlength{\leftmargin}{\cslhangindent}
   \setlength{\itemindent}{-1\cslhangindent}
  \fi
  % set entry spacing
  \setlength{\itemsep}{#2\baselineskip}}}
 {\end{list}}
\usepackage{calc}
\newcommand{\CSLBlock}[1]{\hfill\break\parbox[t]{\linewidth}{\strut\ignorespaces#1\strut}}
\newcommand{\CSLLeftMargin}[1]{\parbox[t]{\csllabelwidth}{\strut#1\strut}}
\newcommand{\CSLRightInline}[1]{\parbox[t]{\linewidth - \csllabelwidth}{\strut#1\strut}}
\newcommand{\CSLIndent}[1]{\hspace{\cslhangindent}#1}

\usepackage{amsthm}
\usepackage{amsmath}
\usepackage{amsfonts}
\usepackage{amssymb}
\usepackage{float}
\usepackage{caption}
\usepackage{subcaption}
% Numbering is tied to sections (e.g., Proposition 2.1, 2.2, etc.)
% Theorems and Propositions (italic)
\theoremstyle{plain}
\newtheorem{theorem}{Theorem}[section]
\newtheorem{proposition}[theorem]{Proposition}
% Definitions and Corollaries (upright)
\theoremstyle{definition}
\newtheorem{definition}{Definition}
\newtheorem{corollary}{Corollary}
\newtheorem{example}{Example}
% \newtheorem{algorithm}[theorem]{Algorithm}
% Optional: formatting for the proof environment
\renewenvironment{proof}
   {\par\noindent\textbf{Proof.}\ }
   {\hfill$\blacksquare$\par}
% \usepackage{algorithm}
% \usepackage{algorithmic}
\usepackage[ruled,vlined,linesnumbered]{algorithm2e}
\makeatletter
\@ifpackageloaded{caption}{}{\usepackage{caption}}
\AtBeginDocument{%
\ifdefined\contentsname
  \renewcommand*\contentsname{Table of contents}
\else
  \newcommand\contentsname{Table of contents}
\fi
\ifdefined\listfigurename
  \renewcommand*\listfigurename{List of Figures}
\else
  \newcommand\listfigurename{List of Figures}
\fi
\ifdefined\listtablename
  \renewcommand*\listtablename{List of Tables}
\else
  \newcommand\listtablename{List of Tables}
\fi
\ifdefined\figurename
  \renewcommand*\figurename{Figure}
\else
  \newcommand\figurename{Figure}
\fi
\ifdefined\tablename
  \renewcommand*\tablename{Table}
\else
  \newcommand\tablename{Table}
\fi
}
\@ifpackageloaded{float}{}{\usepackage{float}}
\floatstyle{ruled}
\@ifundefined{c@chapter}{\newfloat{codelisting}{h}{lop}}{\newfloat{codelisting}{h}{lop}[chapter]}
\floatname{codelisting}{Listing}
\newcommand*\listoflistings{\listof{codelisting}{List of Listings}}
\makeatother
\makeatletter
\makeatother
\makeatletter
\@ifpackageloaded{caption}{}{\usepackage{caption}}
\@ifpackageloaded{subcaption}{}{\usepackage{subcaption}}
\makeatother

\usepackage{bookmark}

\IfFileExists{xurl.sty}{\usepackage{xurl}}{} % add URL line breaks if available
\urlstyle{same} % disable monospaced font for URLs
\hypersetup{
  pdftitle={When Estimation Becomes the Intervention: Multiple Systems Estimation and Causal Inference Under Structural Change},
  pdfauthor={Albert A-N; Scott Moser},
  colorlinks=true,
  linkcolor={black},
  filecolor={Maroon},
  citecolor={RoyalBlue},
  urlcolor={BrickRed},
  pdfcreator={LaTeX via pandoc}}


\title{When Estimation Becomes the Intervention: Multiple Systems
Estimation and Causal Inference Under Structural Change}
\author{Albert A-N \and Scott Moser}
\date{22, July 2025}

\begin{document}
\maketitle

\renewcommand*\contentsname{Table of contents}
{
\hypersetup{linkcolor=}
\setcounter{tocdepth}{3}
\tableofcontents
}

As discussed in the \hyperref[footnote1]{this footnote} the results are
significant.\footnote{Your detailed footnote text goes here.}

\subsection{Introduction}\label{introduction}

\subsubsection{\texorpdfstring{\textbf{1.1
Background}}{1.1 Background}}\label{background}

Multiple Systems Estimation (MSE) is widely used to estimate hidden
population sizes---such as victims of modern slavery---using overlap in
capture across different administrative or NGO lists. In anti-slavery
efforts by organizations like IJM, population estimates inform donor
decisions, intervention targeting, and country-level impact evaluations.

However, interventions often change not just victimization but the
\textbf{way people are recorded}, e.g.~by increasing list overlap
through improved coordination. These measurement artifacts threaten both
MSE's validity and any downstream causal inferences.

\subsubsection{\texorpdfstring{\textbf{1.2 Motivating
Concern}}{1.2 Motivating Concern}}\label{motivating-concern}

If a post-treatment increase in list overlap leads to lower MSE
estimates---despite no true reduction in the population---MSE may
spuriously indicate impact. Conversely, real reductions in prevalence
may be masked by offsetting structural changes in the data.

\begin{quote}
``Without controlling for measurement artifacts (overlap changes),
estimated prevalence changes can mislead causal inference.''
\end{quote}

\subsubsection{\texorpdfstring{\textbf{1.3 Key Assumption Under
Threat}}{1.3 Key Assumption Under Threat}}\label{key-assumption-under-threat}

Causal inference via Difference-in-Differences (DiD) or related methods
assumes \textbf{SUTVA}---that the treatment affects only outcomes, not
how outcomes are measured. When treatment affects \textbf{list
formation, overlap, or coverage}, this assumption fails.

\subsubsection{\texorpdfstring{\textbf{1.4 Prior
Work}}{1.4 Prior Work}}\label{prior-work}

\begin{itemize}
\item
  Lum et al. (\citeproc{ref-lum13-applications}{2013}) and Binette \&
  Steorts (\citeproc{ref-bine22-reliability}{2022}) show that violations
  in list independence or inclusion assumptions cause serious bias in
  MSE estimates.
\item
  Far et al. (\citeproc{ref-far21-multiple}{2021}) show this empirically
  in Romanian anti-slavery MSE work.
\item
  Boesche (\citeproc{ref-boes22-reassessing}{2022}) and Kainou
  (\citeproc{ref-kain17-review}{2017}) critique the fragility of SUTVA
  in quasi-experimental designs, urging alternative assumptions or
  diagnostics.
\end{itemize}

\subsubsection{\texorpdfstring{\textbf{1.5
Contributions}}{1.5 Contributions}}\label{contributions}

\begin{itemize}
\tightlist
\item
  We introduce simulation scenarios showing how structural
  changes---independent of true prevalence---can bias MSE.
\item
  We evaluate four multivariate binary models to simulate capture
  processes with controllable overlap and dependence.
\item
  We relate MSE errors to causal inference threats and offer guidelines
  for quasi-experimental MSE under structural evolution.
\end{itemize}

\subsection{Methodology}\label{methodology}

\subsubsection{\texorpdfstring{\textbf{2.1 Simulation
Framework}}{2.1 Simulation Framework}}\label{simulation-framework}

\begin{itemize}
\tightlist
\item
  \textbf{Population (N):} 1000 hidden individuals
\item
  \textbf{Lists (K):} 3 or 4 (e.g., Police, NGO, Immigration, Medical)
\item
  \textbf{Prevalence Control:} Set true prevalence (e.g., 5\%) and
  simulate list inclusion for those individuals.
\end{itemize}

Each individual is represented by a binary vector (e.g.,
\texttt{{[}1,\ 0,\ 1{]}} means captured on Lists A and C).

\subsubsection{\texorpdfstring{\textbf{2.2 Capture
Models}}{2.2 Capture Models}}\label{capture-models}

We consider four distinct modeling strategies for the joint distribution
over binary list-inclusion vectors
\(\mathbf{Z}_i = (Z_{i1}, \ldots, Z_{iK}) \in \{0,1\}^K\), capturing
various forms of dependence among the \(K\) lists. These include both
exchangeable and non-exchangeable approaches and range from parametric
to nonparametric Bayesian models.

\subsubsection{Log-linear Models (Frequentist \&
Bayesian)}\label{log-linear-models-frequentist-bayesian}

\paragraph{\texorpdfstring{\textbf{Model 1: Log-linear
Model}}{Model 1: Log-linear Model}}\label{model-1-log-linear-model}

\begin{itemize}
\tightlist
\item
  Full model with interaction terms (e.g., 2-list, 3-list overlaps).
\item
  Strength: interpretable, common in MSE literature.
\item
  Limitation: requires large sample sizes; unstable under sparse data.
\end{itemize}

Log-linear models provide a flexible parametric framework for modeling
dependence structures among binary vectors. Let
\(\mathcal{Z} = \{0,1\}^K\) denote the space of binary vectors of length
\(K\). The probability mass function of \(\mathbf{Z}_i \in \mathcal{Z}\)
is modeled as: \[
\mathbb{P}(\mathbf{Z}_i = \mathbf{z}) \propto \exp\left( \sum_{s \subseteq \{1, \ldots, K\}} \theta_s \prod_{k \in s} z_k \right),
\] where the sum is over all subsets \(s\) of \(\{1, \ldots, K\}\), and
\(\theta_s \in \mathbb{R}\) are interaction parameters. For example,
\(\theta_{\{k\}}\) captures the marginal effect of list \(k\), while
\(\theta_{\{k,\ell\}}\) quantifies the pairwise interaction between
lists \(k\) and \(\ell\).

In the Bayesian setting, Gaussian priors
\(\theta_s \sim \mathcal{N}(0, \sigma^2)\) can be placed on the
interaction terms to induce regularization. Due to the exponential
growth in the number of parameters (\(2^K\)), these models are practical
only for small \(K\). Computational inference can proceed via Markov
Chain Monte Carlo (MCMC), often using data augmentation or
Metropolis-Hastings steps.

\subsubsection{Exchangeable Binary Arrays with Finite Sufficient
Statistics}\label{exchangeable-binary-arrays-with-finite-sufficient-statistics}

\paragraph{\texorpdfstring{\textbf{Model 2: Exchangeable Binary
Arrays}}{Model 2: Exchangeable Binary Arrays}}\label{model-2-exchangeable-binary-arrays}

\begin{itemize}
\tightlist
\item
  Captures average dependence between lists without needing all
  high-order terms.
\item
  Strength: lower complexity, works with fewer observations.
\item
  Limitation: doesn't capture specific pairwise list behavior.
\end{itemize}

To mitigate the combinatorial burden of full log-linear modeling, we
consider models where the probability of \(\mathbf{Z}_i\) depends only
on low-dimensional sufficient statistics. Specifically, let: \[
\mathbb{P}(\mathbf{Z}_i = \mathbf{z}) = f\left( \sum_{k=1}^K \alpha_k z_k + \sum_{k<\ell} \beta_{k\ell} z_k z_\ell \right),
\] for some function \(f: \mathbb{R} \rightarrow \mathbb{R}_+\), where
\(\alpha_k\) are main effect parameters and \(\beta_{k\ell}\) encode
symmetric pairwise interactions. This model retains parsimony by
focusing only on low-order interactions and is naturally exchangeable
across individuals \(i\), although not across dimensions \(k\).

Inference proceeds by choosing a parametric or semiparametric form for
\(f(\cdot)\), such as exponential or logistic link functions, and
fitting via MCMC or variational Bayes. This class of models can be
viewed as generalizations of Ising models and belongs to the family of
finite exchangeable binary arrays as discussed in Diaconis \& Freedman
(\citeproc{ref-diac80-finite}{1980}).

\subsubsection{Pólya-Gamma Augmented Multivariate
Probit}\label{puxf3lya-gamma-augmented-multivariate-probit}

\paragraph{\texorpdfstring{\textbf{Model 3: Pólya-Gamma Augmented
Multivariate
Probit}}{Model 3: Pólya-Gamma Augmented Multivariate Probit}}\label{model-3-puxf3lya-gamma-augmented-multivariate-probit}

\begin{itemize}
\tightlist
\item
  Latent Gaussian structure; correlation between lists through shared
  latent trait.
\item
  Strength: smooths estimates, handles moderate \(K\) (lists).
\item
  Limitation: requires more complex inference (e.g., Gibbs sampler).
\end{itemize}

The multivariate probit model introduces correlation across binary
indicators via latent Gaussian variables. Let
\(\mathbf{X}_i = (X_{i1}, \ldots, X_{iK}) \sim \mathcal{N}(\boldsymbol{\mu}, \Sigma)\)
denote a latent Gaussian vector, and define: \[
Z_{ik} = \mathbb{I}(X_{ik} > 0), \quad k = 1, \ldots, K.
\] This induces a joint distribution over \(\mathbf{Z}_i \in \{0,1\}^K\)
where dependence among the \(Z_{ik}\) is encoded entirely in the
covariance matrix \(\Sigma\). A Pólya-Gamma augmentation (Polson et al.
(\citeproc{ref-pols13-bayesian}{2013})) enables efficient Bayesian
inference even for the multivariate probit case by transforming the
probit link into a conditionally Gaussian form, making Gibbs sampling
tractable.

Priors over \(\Sigma\) can be specified using the LKJ distribution for
correlation matrices or inverse-Wishart priors for full covariance
matrices. This model is parsimonious, interpretable, and suitable for
moderate to large \(K\).

\subsubsection{Canonical Bayesian Nonparametric Model: Dirichlet Process
Mixture of Product
Bernoullis}\label{canonical-bayesian-nonparametric-model-dirichlet-process-mixture-of-product-bernoullis}

\paragraph{\texorpdfstring{\textbf{Model 4: Dirichlet Process Mixture of
Product
Bernoullis}}{Model 4: Dirichlet Process Mixture of Product Bernoullis}}\label{model-4-dirichlet-process-mixture-of-product-bernoullis}

\begin{itemize}
\tightlist
\item
  Allows for heterogeneous ``types'' of individuals with different
  capture profiles.
\item
  Strength: flexible, nonparametric.
\item
  Limitation: harder to interpret; computationally expensive.
\end{itemize}

To allow flexible modeling without specifying a fixed number of latent
classes, we employ a Dirichlet Process (DP) mixture model. Let each
individual's binary vector \(\mathbf{Z}_i\) be generated conditionally
independently given a latent parameter vector \(\boldsymbol{\theta}_i\):
\[
Z_{ik} \mid \boldsymbol{\theta}_i \sim \text{Bernoulli}(\theta_{ik}), \quad \boldsymbol{\theta}_i \sim G, \quad G \sim \text{DP}(\alpha, G_0),
\] where \(G_0\) is a base measure over \([0,1]^K\), often taken as a
product of independent Beta distributions:
\(G_0 = \prod_{k=1}^K \text{Beta}(a, b)\).

Marginalizing over \(G\) induces clustering of individuals with similar
list-inclusion profiles. Dependencies among list indicators are induced
through the shared latent parameters \(\boldsymbol{\theta}_i\), even
though the \(Z_{ik}\) are conditionally independent given
\(\boldsymbol{\theta}_i\). Inference is typically performed using
Chinese Restaurant Process (CRP) representations or stick-breaking
constructions.

This model is exchangeable over individuals and highly flexible,
adapting the complexity of the model to the observed data without
requiring a fixed number of latent components.

\subsubsection{\texorpdfstring{\textbf{2.3 Scenario
Definitions}}{2.3 Scenario Definitions}}\label{scenario-definitions}

\paragraph{\texorpdfstring{\textbf{Scenario 1: False
Positive}}{Scenario 1: False Positive}}\label{scenario-1-false-positive}

\begin{itemize}
\item
  \textbf{True N:} 1000 pre, 1000 post.
\item
  \textbf{Overlap Change:} pairwise correlation between lists rises from
  0.0 to 0.6.
\item
  \textbf{Expected Result:} MSE shows drop in estimated N (e.g., from
  980 → 700) despite no true change.
\end{itemize}

\paragraph{\texorpdfstring{\textbf{Scenario 2: False
Negative}}{Scenario 2: False Negative}}\label{scenario-2-false-negative}

\begin{itemize}
\item
  \textbf{True N:} 1000 → 600 (real reduction).
\item
  \textbf{Overlap also increases} (e.g., ρ = 0 → 0.6).
\item
  \textbf{Expected Result:} MSE estimate shows smaller decline (e.g.,
  980 → 720), or possibly none.
\end{itemize}

\paragraph{\texorpdfstring{➕ \textbf{Optional Scenario 3: Structural
Perturbation}}{➕ Optional Scenario 3: Structural Perturbation}}\label{optional-scenario-3-structural-perturbation}

\begin{itemize}
\item
  One list is removed, or a subnational list is treated as national.
\item
  Tests effect of \textbf{coverage misspecification}.
\end{itemize}

\paragraph{\texorpdfstring{➕ \textbf{Optional Scenario 4:
Spillover}}{➕ Optional Scenario 4: Spillover}}\label{optional-scenario-4-spillover}

\begin{itemize}
\item
  Agency coordination in one country causes better overlap in another.
\item
  Tests \textbf{network-based SUTVA violation}.
\end{itemize}

\subsubsection{\texorpdfstring{\textbf{2.4
Metrics}}{2.4 Metrics}}\label{metrics}

\begin{itemize}
\item
  \textbf{Bias:} \(\hat{N} - N\)
\item
  \textbf{RMSE:} Across replications
\item
  \textbf{False positive/negative rate:} In detecting real change
\item
  \textbf{Coverage:} \% of simulations where CI includes true N
\end{itemize}

\subsection{Results}\label{results}

\subsubsection{\texorpdfstring{\textbf{3.1 Scenario
1}}{3.1 Scenario 1}}\label{scenario-1}

\begin{itemize}
\item
  MSE underestimates N when overlap rises.
\item
  Error scales with degree of list correlation.
\item
  Example: At \(\rho = 0.6\), estimate is 25\% too low on average.
\end{itemize}

\subsubsection{\texorpdfstring{\textbf{3.2 Scenario
2}}{3.2 Scenario 2}}\label{scenario-2}

\begin{itemize}
\item
  True prevalence drops, but MSE misses effect due to offsetting
  overlap.
\item
  Estimated declines are \textasciitilde50\% of true magnitude.
\end{itemize}

\subsubsection{\texorpdfstring{\textbf{3.3 Model
Comparisons}}{3.3 Model Comparisons}}\label{model-comparisons}

\begin{itemize}
\item
  Log-linear models perform worst under sparse overlap shifts.
\item
  Pólya-Gamma models more stable, but coverage still poor.
\item
  Dirichlet Process shows best robustness, but least interpretability.
\end{itemize}

\begin{center}\rule{0.5\linewidth}{0.5pt}\end{center}

\subsection{Discussion}\label{discussion}

\subsubsection{\texorpdfstring{\textbf{4.1 Causal Inference with
Structural
Change}}{4.1 Causal Inference with Structural Change}}\label{causal-inference-with-structural-change}

\begin{itemize}
\item
  MSE can't be used for causal inference unless structural list changes
  are controlled.
\item
  SUTVA violations here are \textbf{systematic, not random}---not merely
  noise, but caused by the treatment itself.
\end{itemize}

\subsubsection{\texorpdfstring{\textbf{4.2 Relation to
Literature}}{4.2 Relation to Literature}}\label{relation-to-literature}

\begin{itemize}
\item
  Far et al. (\citeproc{ref-far21-multiple}{2021}) observe this
  empirically in Romania.
\item
  \textbf{Boesche (2020)} proposes SMUTVA (weaker assumptions).
\item
  \textbf{Kazunari (2017)} emphasizes need for \textbf{sensitivity
  analyses}---our simulations fulfill this call.
\end{itemize}

\subsubsection{\texorpdfstring{\textbf{4.3 Practical
Implications}}{4.3 Practical Implications}}\label{practical-implications}

\begin{itemize}
\item
  Never interpret MSE declines as treatment effects without evaluating
  structural list shifts.
\item
  Develop diagnostics for overlap structure (e.g., correlation matrix,
  list dependency graphs).
\item
  Encourage multi-method approaches (e.g., triangulating with survey,
  NSUM).
\end{itemize}

\begin{center}\rule{0.5\linewidth}{0.5pt}\end{center}

\subsection{Conclusion}\label{conclusion}

\begin{itemize}
\item
  MSE estimates are highly sensitive to structural features of list
  overlap.
\item
  Without accounting for SUTVA violations, causal claims from MSE are
  often invalid.
\item
  We call for careful modeling, simulation-based diagnostics, and new
  methodological tools at the intersection of MSE and causal inference.
\end{itemize}

\begin{center}\rule{0.5\linewidth}{0.5pt}\end{center}

\section{Appendices}\label{appendices}

\begin{itemize}
\item
  Formal model definitions
\item
  Code snippets in R/Python
\item
  Convergence diagnostics
\item
  Real-world IJM country scenarios (simulated data)
\end{itemize}

\section*{References}\label{references}
\addcontentsline{toc}{section}{References}

\phantomsection\label{refs}
\begin{CSLReferences}{1}{0}
\bibitem[\citeproctext]{ref-bine22-reliability}
Binette, O., \& Steorts, R. C. (2022). On the reliability of multiple
systems estimation for the quantification of modern slavery.
\emph{Journal of the Royal Statistical Society Series A: Statistics in
Society}, \emph{185}(2), 640--676.
\url{https://doi.org/10.1111/rssa.12803}

\bibitem[\citeproctext]{ref-boes22-reassessing}
Boesche, T. (2022). Reassessing {Quasi-experiments}: {Policy
Evaluation}, {Induction}, and {SUTVA}. \emph{The British Journal for the
Philosophy of Science}, \emph{73}(1), 1--22.
\url{https://doi.org/10.1093/bjps/axz006}

\bibitem[\citeproctext]{ref-diac80-finite}
Diaconis, P., \& Freedman, D. (1980). Finite {Exchangeable Sequences}.
\emph{The Annals of Probability}, \emph{8}(4), 745--764.

\bibitem[\citeproctext]{ref-far21-multiple}
Far, S. S., King, R., Bird, S., Overstall, A., Worthington, H., \&
Jewell, N. (2021). Multiple {Systems Estimation} for {Modern Slavery}:
{Robustness} of {List Omission} and {Combination Special Issue}:
{Applying Multiple Systems Estimation} to {Measure Modern Slavery}:
{Methodological Challenges} and {Innovations}. \emph{Crime and
Delinquency}, \emph{67}(13--14), 2213--2236.

\bibitem[\citeproctext]{ref-kain17-review}
Kainou, K. (2017). \emph{Review of {Necessary Assumptions} for
{Difference-In-Difference} ({DID}) {Estimation} and {Development} of
{Bias Correction Methods} for {DID} where {Spillover Effects} of
{Treatment}/{Causal Effects} to the {Control Group} are not {Ignorable}
and {SUTVA Violated} ({Japanese})} (17075). {Research Institute of
Economy, Trade and Industry (RIETI).}

\bibitem[\citeproctext]{ref-lum13-applications}
Lum, K., Price, M. E., \& Banks, D. (2013). Applications of {Multiple
Systems Estimation} in {Human Rights Research}. \emph{The American
Statistician}, \emph{67}(4), 191--200.
\url{https://doi.org/10.1080/00031305.2013.821093}

\bibitem[\citeproctext]{ref-pols13-bayesian}
Polson, N. G., Scott, J. G., \& Windle, J. (2013). Bayesian {Inference}
for {Logistic Models Using Pólya}--{Gamma Latent Variables}.
\emph{Journal of the American Statistical Association}, \emph{108}(504),
1339--1349. \url{https://doi.org/10.1080/01621459.2013.829001}

\end{CSLReferences}




\end{document}
